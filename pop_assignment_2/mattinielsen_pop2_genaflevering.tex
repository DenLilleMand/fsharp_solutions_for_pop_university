\documentclass[12pt, a4paper, hidelinks]{article}

% Packages:
\usepackage{graphicx}                   % For figure includes
\usepackage[T1]{fontenc}                % For mixing up \textsc{} with \textbf{}
\usepackage[utf8]{inputenc}             % For scandinavian input characters(æøå)
\usepackage{amsfonts, amsmath, amssymb} % For common mathsymbols and fonts
\usepackage[danish]{babel}              % For danish titles
\usepackage{hyperref}                   % For making links and refrences
\usepackage{url}                        % Just because {~_^}
\usepackage{array}                      % ...
\usepackage[usenames, dvipsnames, svgnames, table]{xcolor}
\usepackage{tabularx, colortbl}
\usepackage{verbatim} % For entering code snippets.
\usepackage{fancyvrb} % A "fancy" verbatim (for pseudo code).
\usepackage{listings} % For boxed codesnippets, and file includes. (begin)
\usepackage{lipsum}   % For generating dummy text at this demonstration
\usepackage{float}

% Basic layout:
\setlength{\textwidth}{165mm}
\setlength{\textheight}{240mm}
\setlength{\parindent}{0mm}
\setlength{\parskip}{\parsep}
\setlength{\headheight}{0mm}
\setlength{\headsep}{0mm}
\setlength{\hoffset}{-2.5mm}
\setlength{\voffset}{0mm}
\setlength{\footskip}{15mm}
\setlength{\oddsidemargin}{0mm}
\setlength{\topmargin}{0mm}
\setlength{\evensidemargin}{0mm}


\newcolumntype{C}[1]{>{\centering\arraybackslash}p{#1}}

% Colors:
\definecolor{KU-red}{RGB}{144, 26, 30}

% Text Coloring:
\newcommand{\green}[1]{\textbf{\color{green}{#1}}}
\newcommand{\blue} [1]{\textbf{\color{blue} {#1}}}
\newcommand{\red}  [1]{\textbf{\color{red}  {#1}}}


% ************************* Start Document *****************
\begin{document}

% ************************* Page Header ********************
\begin{minipage}[b]{1.0\linewidth}
\includegraphics[height=30mm]{bilag/KULogo}

\vspace*{-16ex}
\begin{center}
    {\Large \bf Programmering og Problemløsning} \vspace*{1ex} \\
    {\large Ugeopgave 2} \vspace*{1ex} \\
    {\large Matti Andreas Nielsen  } \\
    {\large \today{}  }
\end{center}
\vspace*{-3pt}
{\color{KU-red}\hrule}
\end{minipage}
\vspace{2ex}

% **************** Assignment Starts Here ******************
\tableofcontents \newpage

\section{2i.0 }

Mine 3 udtryk:\\
1. StringLiteral, operator, stringLiteral (* no space*)\\
2. expression, stringLiteral (* no space *)\\
3. stringLiteral\\

Eksempler i samme rækkefølge som udtrykne:\\
1. "U+0041"+"U+0042"\\
2. "U+0055"+"U+0056""U+0057""U+0058"\\
3. "U+0067""U+0068"\\


Eksempler som ikke er gyldige er: \\
1. ++  \\
2. "dfjgdfk232323""U+0042"\\


\section{2i.1}
Udfyld følgende tabel:\\

\begin{tabular}{l*{6}{c}r}
Decimal & Binær & Heximal & Oktal \\
\hline
10 &  & &     \\
& 10101 & &  \\
& & 3f &      \\
& & & 77      \\
\end{tabular}

Fra decimal til binær kan vi udregne det ved at regne n til 2, sådan at man f.eks udregner $$ 2^0 = 1 \text{ }, 2^1 = 2 \text{ }, 2^2 = 4 \text{ },   2^3 = 8 \text{ },   2^4 = 16 $$
så tager man det højeste tal, som ikke går over det tal vi skal finde,
i det her tilfælde prøver jeg at omregne 10 til binær. 
Så det er 8, og så siger vi 8 + det næste tal der ikke får os over 10, i det her tilfælde 2. Så når vi er fundet frem til 8 + 2 så konvertere vi de pladser direkte til binært fordi vi ved at 8 er plads 4 i binær og 2 er plads 2, man kan se det endelige svar i tabellen.
Vi ville også kunne have brugt division med 2 metoden, hvor man bliver ved med at dividere med 2, så hvis vi skulle finde 10 med den metode, ville vi kunne sige.

$$ 10 \bmod 5  = 5 \text{(0 i mente)} $$
$$ 5 \bmod 2 = 2 \text{(1 i mente)} $$
$$ 2 \bmod 2  = 1 \text{(0 i mente)} $$
$$ 2 \bmod 1 = 0 \text{(1 i mente)} $$

Og så flipper man den rundt, og siger at den øverste er den mindste bit, og den nederste er den  højeste det giver os 1010 hvilket den anden metode også gav os.

For at udregne binært til Decimal bruger vi bare en helt almindelig konverterings tabel, vi kan prøve med det binære 10101:

\begin{tabular}{l*{8}{c}r}
Decimal værdi & 256 & 128 & 64 & 32 & 16 & 8 & 4 & 2 & 1 \\
Binær værdi   & 0   & 0   &  0 &  0 &  1 & 0 & 1 & 0 & 1  \\
\end{tabular}


Binær til hex:
Hex går op i 16, det skal vi bruge 4 bits til at repræsentere i binært. Man skal opdele vores binære tal i bloks af 4 bits af gangen, og så ligge dem sammen. 
10101  -->  0001 + 0101  --> 1 + 5 = 15

Konvertering fra binært til oktal er næsten det samme som binært til hex. Istedet for at opdele tallet i blokke af 4, deler vi nu det binære tal op i blokke af 3, da der kun skal bruges 3 bit for at repræsentere et enkelt cifre i oktal.

\begin{tabular}{l*{6}{c}r}
Decimal & Binær & Heximal & Oktal \\
\hline
10 & 1010  & A & 12\\
21 & 10101 & 15 & 25 \\
63 & 111111 & 3f & 77 \\
63 & 111111 & 3f & 77 \\
\end{tabular}

\section{2i.2}
Kig i bilaget Opgave2i16.fsx

\end{document}
