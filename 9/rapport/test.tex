Mine tests kan alle sammen findes i CarTests.fsx, jeg har prøvet at teste således at jeg både
har testet med korrekte handlinger og ukorrekte handlinger og de cases som
er på grænsen imellem de to.
Jeg har annoteret alle mine tests med en custom attribute, som man kan se på figur~\ref{fig:TestAnnotationAttribute}
\begin{figure}[!htb]
  \lstset{language=FSharp}
  \text{
    \lstinputlisting[firstline=9,lastline=10]{../TestLib.fs}
  }
  \caption{Custom test annotation type}
\label{fig:TestAnnotationAttribute}
\end{figure}

og man kan se hvordan jeg bruger annotationen i figur~\ref{fig:AnnotationInAction}
\begin{figure}[!htb]
  \lstset{language=FSharp}
  \text{
    \lstinputlisting[firstline=12,lastline=19]{../CarTests.fsx}
  }
  \caption{Custom test annotation brugt}
\label{fig:AnnotationInAction}
\end{figure}

Så når jeg skal kører mine tests behøver jeg ikke at sætte dem alle sammen op i en klasse
for sig selv og kører dem, men istedet kan jeg, som man kan se i figur~\ref{fig:Reflection}
\begin{figure}[!htb]
  \lstset{language=FSharp}
  \text{
    \lstinputlisting[firstline=7,lastline=20]{../TestRunner.fsx}
  }
  \caption{Reflection og Annotations i aktion}
\label{fig:Reflection}
\end{figure}
bare søge igennem alle typerne jeg har loaded. Hvorefter jeg kan finde alle de metoder der har min test annotation og eksekverer dem.
Noget af det jeg mangler for at gøre dét rigtig snedigt, er at fjerne konsol output og andre ting som intet har med assertions at gøre,
fra mit assertions bibliotek. Så ville man skulle lave min testrunner om sådan at den bruger testens returnerede boolean til at 
outputte om testen gik godt eller dårligt, men også overordnet hvor mange tests der gik godt ud af et totalt antal.
Via reflection ville man endda kunne bruge navnet på testen i ens output sådan at jeg heller ikke behøver 
at give variablen testname ind til min output metode, som virker lidt redundant når selvsamme information ligger i metodens navn.

