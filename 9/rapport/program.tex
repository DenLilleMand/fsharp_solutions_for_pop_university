Man kan eksekvere programmet ved at kører den default make case i makefilen,
som automatisk bliver kørt hvis man kører make uden nogle argumenter.
\begin{lstlisting}
    make
\end{lstlisting}
Man vil dog ikke få noget output,
som beskrevet i opgaven fordi jeg synes at dé ting man bliver bedt om at skrive ud i konsollen,
hænger bedre sammen med unit tests, så dem har jeg prøvet at lave lidt flere af istedet.
Man kører tests ved at kører make med tests som argument.
\begin{lstlisting}
    make tests
\end{lstlisting}

\section{Programmet}
Hele programmet befinder sig i filen Car.fsx, der har fire member attributter
som vi kan se i figur~\ref{fig:MemberAttributesInCar}.
\begin{figure}[!htb]
  \lstset{language=FSharp}
  \text{
    \lstinputlisting[firstline=17,lastline=36]{../Car.fsx}
  }
  \caption{Members/Attributter i Car}
\label{fig:MemberAttributesInCar}
\end{figure}

Members i FSharp har public accessibility per default, så dét er den accessibility dé har
fået og det giver god mening i forhold til min testing lige nu,
sådan at jeg ikke skal bruge reflection til at ændre accessibility.
Dog hvis man legede at dét var en rigtig car klasse,
burde man selvfølgelig lave sådan noget, som benzin,
fart og yearOfModel private, og hvor de kun kan ændres igennem et meget specifikt API.

Opgaven nævner også 

\begin{lstlisting}
    gasLeft
\end{lstlisting}

 og 

\begin{lstlisting}
    addGas
\end{lstlisting}

hvor jeg har taget et design valg
og bare lavet dét til standard getters og setters på benzin attributten,
fordi de udfylder præcis den samme funktionalitet.

Så har jeg de to metoder i klassen, som man kan se på figur~\ref{fig:MethodsInCar}
\begin{figure}[!htb]
  \lstset{language=FSharp}
  \text{
    \lstinputlisting[firstline=43,lastline=60]{../Car.fsx}
  }
  \caption{Funktionerne i Car}
\label{fig:MethodsInCar}
\end{figure}

Funktionaliteten i Accelerate sætter farten op på bilen og reducere vores benzin,
dette er muligt
indtil benzinen er ved at ramme 0, hvorefter vi simulere en reducering af resten
af benzinen og farten der begge falder til 0, hvorefter vi smider en exception.
